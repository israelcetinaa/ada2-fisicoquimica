\documentclass[12pt]{article}
\usepackage{amsmath,amssymb}
\usepackage{amsmath, array}
\usepackage{amsmath,amsthm, amssymb, latexsym}
\usepackage{mathtools}
\usepackage[breaklinks=true]{hyperref}
\hypersetup{% 
	pdfborder = {0 0 0} 
} 
\usepackage{graphicx}
\usepackage{float}
\usepackage[spanish]{babel}
\usepackage[utf8]{inputenc}
\usepackage{multirow}
\usepackage{enumerate}
\usepackage{setspace}
\usepackage{subfigure}


\begin{document}
	\begin{titlepage}
		
		\begin{center}
			\vspace*{-1in}
			\begin{figure}[htb]
				\begin{center}
					\includegraphics[width=8cm]{logo}
				\end{center}
			\end{figure}
			
			Facultad de Ingeniería\\
			\vspace*{0.15in}
			Licenciatura en Ingeniería Física \\
			\vspace*{0.2in}
			\begin{large}
				Alan Mosqueda Camacho\\
				Carmen Andrea Rivera Martínez\\
				Francisco Abimael Yam Hong\\
				Gonzalo Herrera Ramirez\\
				Jesús Alejandro Salazar González\\
				José Israel Cetina Palomo\\
				Pedro Felipe Baeza Ortiz\\
			\end{large}
			\vspace*{0.2in}
			\begin{Large}
				\textbf{ADA 2: Ejercicios} \\
			\end{Large}
			\vspace*{0.2in}
			\begin{large}
				Fisicoquímica \\
			\end{large}
			\vspace*{0.3in}
			\rule{80mm}{0.1mm}\\
			\vspace*{0.1in}
			\begin{large}
				Maestro: Avel Adolfo González Sánchez\\
			\end{large}
		\end{center}
		
	\end{titlepage}
\section*{Problema 7-11}
Un mil de gas ideal a $27^{\circ}\mathrm{C}$ y $10 \; \mathrm{atm}$, se expande adiabáticamente hasta una presión constante opositora de $1\;\mathrm{atm}$. calcular la temperatura final, $\mathrm{Q}$, $\mathrm{W}$, $\Delta\mathrm{E}$ y $\Delta\mathrm{H}$ para los dos casos, $\mathrm{\bar{c}}_v=\frac{3\mathrm{R}}{2}$, $\mathrm{\bar{c}}_v=\frac{5\mathrm{R}}{2}$.\\

Primero el caso de $\mathrm{\bar{c}}_v=\frac{3\mathrm{R}}{2}$:
\begin{center}
	\begin{tabular}{c c c}
		Estado 1 &   &  Estado 2 \\
		$\mathrm{T}_1=27^{\circ}\mathrm{C}$\; (300.15$\mathrm{K}$) &   & $\mathrm{T}_2=?$ \\
		$\mathrm{P}_1=10\mathrm{atm}$ &   & $\mathrm{P}_2=1\mathrm{atm}$ \\
		$\mathrm{V}_1=?$ &   & $\mathrm{V}_2=?$\\
		$n=1\mathrm{mol}$&   & $n=1\mathrm{mol}$
	\end{tabular}
\begin{displaymath}
	\mathrm{R}=0.08206\; \frac{\mathrm{atm}\cdot\mathrm{L}}{\mathrm{mol}\cdot\mathrm{K}}
\end{displaymath}
\end{center}

Calculando el volumen en el estado 1

\begin{displaymath}
	\mathrm{P}\mathrm{V}=n\mathrm{R}\mathrm{T} \rightarrow \mathrm{P}_1\mathrm{V}_1=n\mathrm{R}\mathrm{T}_1 \rightarrow \mathrm{V}_1=\frac{n\mathrm{R}\mathrm{T}_1}{\mathrm{P}_1}
\end{displaymath}

\begin{displaymath}
	\mathrm{V}_1=\frac{(1\mathrm{mol})(	\mathrm{R}=0.08206\; \frac{\mathrm{atm}\cdot\mathrm{L}}{\mathrm{mol}\cdot\mathrm{K}})(300.15\mathrm{K})}{10\mathrm{atm}}
\end{displaymath}
\begin{displaymath}
	\therefore\mathrm{V}_1=2.0463 \;\mathrm{L}
\end{displaymath}

Para el caso no reversible procedemos de la siguiente forma, tenemos dos incognitas: $\mathrm{T}_2$ y $\mathrm{V}_2$. Se necesitan dos ecuaciones, la primera ecuación es:
\begin{displaymath}
	\mathrm{P}_2\mathrm{V}_2=n\mathrm{R}\mathrm{T}_2 \rightarrow \mathrm{V}_2=\frac{n\mathrm{R}\mathrm{T}_2}{\mathrm{P}_2}
\end{displaymath}

La siguiente ecuación es:
\begin{displaymath}
	\mathrm{P}\Delta\mathrm{V}=-\mathrm{W} \rightarrow 	\mathrm{P}\Delta\mathrm{V}= -n\mathrm{\bar{c}}_v\Delta\mathrm{T}
\end{displaymath}
Que sería igual a:
\begin{displaymath}
	\mathrm{P}_2(\mathrm{V}_2-\mathrm{V}_1)=-n\mathrm{\bar{c}}_v(\mathrm{T}_2-\mathrm{T}_1)
\end{displaymath}
Desarrollando y sustituyendo $\mathrm{V}_2$ para despejar $\mathrm{T}_2$

\begin{displaymath}
	n\mathrm{\bar{c}}_v\mathrm{T}_1+\mathrm{P}_2\mathrm{V}_1=\mathrm{T}_2\left[ n\mathrm{\bar{c}}_v+n\mathrm{R} \right]
\end{displaymath}
Agregando el valor de $\mathrm{\bar{c}}_v$ y terminando el despeje
\begin{displaymath}
	\mathrm{T}_2=\frac{2}{5\mathrm{R}n}\left[ \frac{3\mathrm{R}n\mathrm{T}_1}{2}+\mathrm{P}_2\mathrm{V}_1 \right]
\end{displaymath}
\begin{displaymath}
	\Rightarrow	\mathrm{T}=\frac{2}{5}\left[ \frac{3\mathrm{T}}{2}+\frac{\mathrm{P}_2\mathrm{V}_1}{\mathrm{R}n} \right]
\end{displaymath}

Sustituyendo valores y resolviendo
\begin{displaymath}
	\mathrm{T}_2=\frac{2}{5} \left[ \frac{3(300.15\mathrm{K})}{2}+\frac{(1\mathrm{atm})(2.463\mathrm{L})}{(0.08206\; \frac{\mathrm{atm}\cdot\mathrm{L}}{\mathrm{mol}\cdot\mathrm{K}})(1\mathrm{mol})} \right]
\end{displaymath}
\begin{displaymath}
	\therefore \mathrm{T}_2=192.1\mathrm{K}
\end{displaymath}
Al ser un proceso adiabático 
\begin{displaymath}
	\mathrm{Q}=0
\end{displaymath}
Calculando $\Delta\mathrm{E}$
\begin{displaymath}
	\Delta \mathrm{E}=\mathrm{\bar{c}}_v(\mathrm{T}_2-\mathrm{T}_1)\rightarrow \Delta\mathrm{E}=\frac{3\mathrm{R}}{2}\left( \mathrm{T}_2-\mathrm{T}_1 \right)
\end{displaymath}
\begin{displaymath}
	\Rightarrow \Delta\mathrm{E}=\frac{3}{2}\left( 0.08206\; \frac{\mathrm{atm}\cdot\mathrm{L}}{\mathrm{mol}\cdot\mathrm{K}} \right) \left( 192.1-300.15 \right)\mathrm{K}
\end{displaymath}
\begin{displaymath}
	\therefore \Delta\mathrm{E}=-1.35\frac{\mathrm{kJ}}{\mathrm{mol}}
\end{displaymath}
Calculando $\mathrm{W}$
\begin{displaymath}
	-\mathrm{W}=\Delta\mathrm{E}\Rightarrow \mathrm{W}=1.35\frac{\mathrm{kJ}}{\mathrm{mol}}
\end{displaymath}
Calculando $\Delta\mathrm{H}$
\begin{displaymath}
	\Delta\mathrm{H}=(\mathrm{\bar{c}}_v+\mathrm{R})(\mathrm{T}_2-\mathrm{T}_1)\Rightarrow \Delta\mathrm{H}=\frac{5\mathrm{R}}{2}(\mathrm{T}_2-\mathrm{T}_1)
\end{displaymath}
\begin{displaymath}
	\Delta\mathrm{H}=\frac{5}{2}\left( 0.08206\; \frac{\mathrm{atm}\cdot\mathrm{L}}{\mathrm{mol}\cdot\mathrm{K}} \right)\left( 192.1-300.15 \right)\mathrm{K}
\end{displaymath}
\begin{displaymath}
	\therefore \Delta\mathrm{H}=-2.24 \frac{\mathrm{kJ}}{\mathrm{mol}}
\end{displaymath}
\newpage
Caso 2 con $\mathrm{\bar{c}}_v=\frac{5\mathrm{R}}{2}$ y no reversible.\\
\\
El volumen $\mathrm{V}_1$ se mantiene igual, por lo que
\begin{displaymath}
	\mathrm{V}_12.463\mathrm{L}
\end{displaymath}
Calculando $\mathrm{T}_2$ usando la fórmula que ya se había deducido y sustituyendo el nuevo valor de $\mathrm{\bar{c}}_v$ se tiene:
\begin{displaymath}
	\mathrm{T}_2=\frac{2}{7}\left[ \frac{5\mathrm{T}_1}{2}+\frac{\mathrm{P}_2\mathrm{V}_1}{\mathrm{R}n} \right]
\end{displaymath}

\begin{displaymath}
	\mathrm{T}_2=\frac{2}{7}\left[ \frac{5(300.15\mathrm{K})}{2} +\frac{(1\mathrm{atm})(2.463\mathrm{L})}{(0.08206\; \frac{\mathrm{atm}\cdot\mathrm{L}}{\mathrm{mol}\cdot\mathrm{K}})(1\mathrm{mol})} \right]
\end{displaymath}

\begin{displaymath}
	\therefore \mathrm{T}_2=223\mathrm{K}
\end{displaymath}
Al ser un proceso adiabático 
\begin{displaymath}
	\mathrm{Q}=0
\end{displaymath}
De igual forma, los cálculos de $\mathrm{W}$, $\Delta\mathrm{E}$ y $\Delta\mathrm{H}$ son identicos, solo cambiando el valor de $\mathrm{\bar{c}}_v$, por lo que obviaremos el desarrollo para pasar directamente al valor obtenido.

\begin{displaymath}
	\Delta\mathrm{E}=-1.6\; \frac{\mathrm{kJ}}{\mathrm{mol}}
\end{displaymath}

\begin{displaymath}
	\Delta\mathrm{W}=1.6\; \frac{\mathrm{kJ}}{\mathrm{mol}}
\end{displaymath}

\begin{displaymath}
	\Delta\mathrm{H}=-2.24\; \frac{\mathrm{kJ}}{\mathrm{mol}}
\end{displaymath}
















\end{document}