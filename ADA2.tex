\documentclass[12pt]{article}
\usepackage{amsmath,amssymb}
\usepackage{amsmath, array}
\usepackage{amsmath,amsthm, amssymb, latexsym}
\usepackage{mathtools}
\usepackage[breaklinks=true]{hyperref}
\hypersetup{% 
	pdfborder = {0 0 0} 
} 
\usepackage{graphicx}
\usepackage{float}
\usepackage[spanish]{babel}
\usepackage[utf8]{inputenc}
\usepackage{multirow}
\usepackage{enumerate}
\usepackage{setspace}
\usepackage{subfigure}


\begin{document}
	\begin{titlepage}
		
		\begin{center}
			\vspace*{-1in}
			\begin{figure}[htb]
				\begin{center}
					\includegraphics[width=8cm]{logo}
				\end{center}
			\end{figure}
			
			Facultad de Ingeniería\\
			\vspace*{0.15in}
			Licenciatura en Ingeniería Física \\
			\vspace*{0.2in}
			\begin{large}
				Alan Mosqueda Camacho\\
				Carmen Andrea Rivera Martínez\\
				Francisco Abimael Yam Hong\\
				Gonzalo Herrera Ramirez\\
				Jesús Alejandro Salazar González\\
				José Israel Cetina Palomo\\
				Pedro Felipe Baeza Ortiz\\
			\end{large}
			\vspace*{0.2in}
			\begin{Large}
				\textbf{ADA 2: Ejercicios} \\
			\end{Large}
			\vspace*{0.2in}
			\begin{large}
				Fisicoquímica \\
			\end{large}
			\vspace*{0.3in}
			\rule{80mm}{0.1mm}\\
			\vspace*{0.1in}
			\begin{large}
				Maestro: Avel Adolfo González Sánchez\\
			\end{large}
		\end{center}
		
	\end{titlepage}
\section*{Problema 7-11}
Un mil de gas ideal a $27^{\circ}\mathrm{C}$ y $10 \; \mathrm{atm}$, se expande adiabáticamente hasta una presión constante opositora de $1\;\mathrm{atm}$. calcular la temperatura final, $\mathrm{Q}$, $\mathrm{W}$, $\Delta\mathrm{E}$ y $\Delta\mathrm{H}$ para los dos casos, $\mathrm{\bar{c}}_v=\frac{3\mathrm{R}}{2}$, $\mathrm{\bar{c}}_v=\frac{5\mathrm{R}}{2}$.\\

Primero el caso de $\mathrm{\bar{c}}_v=\frac{3\mathrm{R}}{2}$:
\begin{center}
	\begin{tabular}{| c | c |}
		\hline
		Estado 1 &     Estado 2 \\ \hline
		$\mathrm{T}_1=27^{\circ}\mathrm{C}$\; (300.15$\mathrm{K}$) &    $\mathrm{T}_2=?$ \\
		$\mathrm{P}_1=10\mathrm{atm}$ &    $\mathrm{P}_2=1\mathrm{atm}$ \\
		$\mathrm{V}_1=?$ &    $\mathrm{V}_2=?$\\
		$n=1\mathrm{mol}$&    $n=1\mathrm{mol}$\\ \hline
	\end{tabular}

\begin{displaymath}
	\mathrm{R}=0.08206\; \frac{\mathrm{atm}\cdot\mathrm{L}}{\mathrm{mol}\cdot\mathrm{K}}
\end{displaymath}
\end{center}

Calculando el volumen en el estado 1

\begin{displaymath}
	\mathrm{P}\mathrm{V}=n\mathrm{R}\mathrm{T} \rightarrow \mathrm{P}_1\mathrm{V}_1=n\mathrm{R}\mathrm{T}_1 \rightarrow \mathrm{V}_1=\frac{n\mathrm{R}\mathrm{T}_1}{\mathrm{P}_1}
\end{displaymath}

\begin{displaymath}
	\mathrm{V}_1=\frac{(1\mathrm{mol})(	\mathrm{R}=0.08206\; \frac{\mathrm{atm}\cdot\mathrm{L}}{\mathrm{mol}\cdot\mathrm{K}})(300.15\mathrm{K})}{10\mathrm{atm}}
\end{displaymath}

\begin{displaymath}
	\therefore\mathrm{V}_1=2.0463 \;\mathrm{L}
\end{displaymath}

Para el caso no reversible procedemos de la siguiente forma, tenemos dos incognitas: $\mathrm{T}_2$ y $\mathrm{V}_2$. Se necesitan dos ecuaciones, la primera ecuación es:

\begin{displaymath}
	\mathrm{P}_2\mathrm{V}_2=n\mathrm{R}\mathrm{T}_2 \rightarrow \mathrm{V}_2=\frac{n\mathrm{R}\mathrm{T}_2}{\mathrm{P}_2}
\end{displaymath}

La siguiente ecuación es:

\begin{displaymath}
	\mathrm{P}\Delta\mathrm{V}=-\mathrm{W} \rightarrow 	\mathrm{P}\Delta\mathrm{V}= -n\mathrm{\bar{c}}_v\Delta\mathrm{T}
\end{displaymath}

Que sería igual a:

\begin{displaymath}
	\mathrm{P}_2(\mathrm{V}_2-\mathrm{V}_1)=-n\mathrm{\bar{c}}_v(\mathrm{T}_2-\mathrm{T}_1)
\end{displaymath}

Desarrollando y sustituyendo $\mathrm{V}_2$ para despejar $\mathrm{T}_2$

\begin{displaymath}
	n\mathrm{\bar{c}}_v\mathrm{T}_1+\mathrm{P}_2\mathrm{V}_1=\mathrm{T}_2\left[ n\mathrm{\bar{c}}_v+n\mathrm{R} \right]
\end{displaymath}

Agregando el valor de $\mathrm{\bar{c}}_v$ y terminando el despeje

\begin{displaymath}
	\mathrm{T}_2=\frac{2}{5\mathrm{R}n}\left[ \frac{3\mathrm{R}n\mathrm{T}_1}{2}+\mathrm{P}_2\mathrm{V}_1 \right]
\end{displaymath}

\begin{displaymath}
	\Rightarrow	\mathrm{T}=\frac{2}{5}\left[ \frac{3\mathrm{T}}{2}+\frac{\mathrm{P}_2\mathrm{V}_1}{\mathrm{R}n} \right]
\end{displaymath}

Sustituyendo valores y resolviendo

\begin{displaymath}
	\mathrm{T}_2=\frac{2}{5} \left[ \frac{3(300.15\mathrm{K})}{2}+\frac{(1\mathrm{atm})(2.463\mathrm{L})}{(0.08206\; \frac{\mathrm{atm}\cdot\mathrm{L}}{\mathrm{mol}\cdot\mathrm{K}})(1\mathrm{mol})} \right]
\end{displaymath}

\begin{displaymath}
	\therefore \mathrm{T}_2=192.1\mathrm{K}
\end{displaymath}

Al ser un proceso adiabático 

\begin{displaymath}
	\mathrm{Q}=0
\end{displaymath}

Calculando $\Delta\mathrm{E}$

\begin{displaymath}
	\Delta \mathrm{E}=\mathrm{\bar{c}}_v(\mathrm{T}_2-\mathrm{T}_1)\rightarrow \Delta\mathrm{E}=\frac{3\mathrm{R}}{2}\left( \mathrm{T}_2-\mathrm{T}_1 \right)
\end{displaymath}

\begin{displaymath}
	\Rightarrow \Delta\mathrm{E}=\frac{3}{2}\left( 0.08206\; \frac{\mathrm{atm}\cdot\mathrm{L}}{\mathrm{mol}\cdot\mathrm{K}} \right) \left( 192.1-300.15 \right)\mathrm{K}
\end{displaymath}

\begin{displaymath}
	\therefore \Delta\mathrm{E}=-1.35\frac{\mathrm{kJ}}{\mathrm{mol}}
\end{displaymath}

Calculando $\mathrm{W}$

\begin{displaymath}
	-\mathrm{W}=\Delta\mathrm{E}\Rightarrow \mathrm{W}=1.35\frac{\mathrm{kJ}}{\mathrm{mol}}
\end{displaymath}

Calculando $\Delta\mathrm{H}$

\begin{displaymath}
	\Delta\mathrm{H}=(\mathrm{\bar{c}}_v+\mathrm{R})(\mathrm{T}_2-\mathrm{T}_1)\Rightarrow \Delta\mathrm{H}=\frac{5\mathrm{R}}{2}(\mathrm{T}_2-\mathrm{T}_1)
\end{displaymath}

\begin{displaymath}
	\Delta\mathrm{H}=\frac{5}{2}\left( 0.08206\; \frac{\mathrm{atm}\cdot\mathrm{L}}{\mathrm{mol}\cdot\mathrm{K}} \right)\left( 192.1-300.15 \right)\mathrm{K}
\end{displaymath}

\begin{displaymath}
	\therefore \Delta\mathrm{H}=-2.24 \frac{\mathrm{kJ}}{\mathrm{mol}}
\end{displaymath}

Caso 2 con $\mathrm{\bar{c}}_v=\frac{5\mathrm{R}}{2}$ y no reversible.\\
\\
El volumen $\mathrm{V}_1$ se mantiene igual, por lo que
\begin{displaymath}
	\mathrm{V}_12.463\mathrm{L}
\end{displaymath}
Calculando $\mathrm{T}_2$ usando la fórmula que ya se había deducido y sustituyendo el nuevo valor de $\mathrm{\bar{c}}_v$ se tiene:
\begin{displaymath}
	\mathrm{T}_2=\frac{2}{7}\left[ \frac{5\mathrm{T}_1}{2}+\frac{\mathrm{P}_2\mathrm{V}_1}{\mathrm{R}n} \right]
\end{displaymath}

\begin{displaymath}
	\mathrm{T}_2=\frac{2}{7}\left[ \frac{5(300.15\mathrm{K})}{2} +\frac{(1\mathrm{atm})(2.463\mathrm{L})}{(0.08206\; \frac{\mathrm{atm}\cdot\mathrm{L}}{\mathrm{mol}\cdot\mathrm{K}})(1\mathrm{mol})} \right]
\end{displaymath}

\begin{displaymath}
	\therefore \mathrm{T}_2=223\mathrm{K}
\end{displaymath}
Al ser un proceso adiabático 
\begin{displaymath}
	\mathrm{Q}=0
\end{displaymath}
De igual forma, los cálculos de $\mathrm{W}$, $\Delta\mathrm{E}$ y $\Delta\mathrm{H}$ son identicos, solo cambiando el valor de $\mathrm{\bar{c}}_v$, por lo que obviaremos el desarrollo para pasar directamente al valor obtenido.

\begin{displaymath}
	\Delta\mathrm{E}=-1.6\; \frac{\mathrm{kJ}}{\mathrm{mol}}
\end{displaymath}

\begin{displaymath}
	\Delta\mathrm{W}=1.6\; \frac{\mathrm{kJ}}{\mathrm{mol}}
\end{displaymath}

\begin{displaymath}
	\Delta\mathrm{H}=-2.24\; \frac{\mathrm{kJ}}{\mathrm{mol}}
\end{displaymath}
\newpage
\section*{Problema 7-12}
Repetir el problema 7-11 suponiendo que la expansión es reversible

Primero el caso de $\mathrm{\bar{c}}_v=\frac{3\mathrm{R}}{2}$:
\begin{center}
	\begin{tabular}{| c | c |}
		\hline
		Estado 1 &     Estado 2 \\ \hline
		$\mathrm{T}_1=27^{\circ}\mathrm{C}$\; (300.15$\mathrm{K}$) &    $\mathrm{T}_2=?$ \\
		$\mathrm{P}_1=10\mathrm{atm}$ &    $\mathrm{P}_2=1\mathrm{atm}$ \\
		$\mathrm{V}_1=?$ &    $\mathrm{V}_2=?$\\
		$n=1\mathrm{mol}$&    $n=1\mathrm{mol}$\\ \hline
	\end{tabular}
	\begin{displaymath}
		\mathrm{R}=0.08206\; \frac{\mathrm{atm}\cdot\mathrm{L}}{\mathrm{mol}\cdot\mathrm{K}}
	\end{displaymath}
\end{center}

Calculando el volumen 1

\begin{displaymath}
	\mathrm{P}_1\mathrm{V}_1=n\mathrm{R}\mathrm{T}_1\Rightarrow \mathrm{V}_1=\frac{n\mathrm{R}\mathrm{T}_1}{\mathrm{P}_1}
\end{displaymath}

Sustituyendo

\begin{displaymath}
	\mathrm{V}_1=\frac{\left( 1\mathrm{mol} \right) \left( 0.08206\; \frac{\mathrm{atm}\cdot\mathrm{L}}{\mathrm{mol}\cdot\mathrm{K}} \right)\left( 300.15\mathrm{K} \right)}{10\mathrm{atm}}
\end{displaymath}

\begin{displaymath}
	\therefore\mathrm{V}_1=2.463\mathrm{L}
\end{displaymath}

despejando $\mathrm{V}_2$ de la siguiente ecuación 

\begin{displaymath}
	\mathrm{P}_2\mathrm{V}_2=n\mathrm{R}\mathrm{T}_2\Rightarrow \mathrm{V}_2=\frac{n\mathrm{R}\mathrm{T}_2}{\mathrm{P}_2}
\end{displaymath}

Ahora para calcular $\mathrm{T}_2$ usamos la siguiente ecuación:

\begin{displaymath}
	\mathrm{V}_1(\mathrm{T}_1)^{\frac{\mathrm{\bar{c}}_v}{\mathrm{R}}}= \mathrm{V}_2(\mathrm{T}_2)^{\frac{\mathrm{\bar{c}}_v}{\mathrm{R}}}
\end{displaymath}

Sustituyendo $\mathrm{V}_2$ en la ecuación anterior

\begin{displaymath}
	\mathrm{V}_1(\mathrm{T}_1)^{\frac{\mathrm{\bar{c}}_v}{\mathrm{R}}}=\frac{n\mathrm{R}\mathrm{T}_2}{\mathrm{P}_2}(\mathrm{T}_2)^{\frac{\mathrm{\bar{c}}_v}{\mathrm{R}}}
\end{displaymath}

\begin{displaymath}
	\frac{\mathrm{P}_2\mathrm{V}_1(\mathrm{T}_1)^{\frac{\mathrm{\bar{c}}_v}{\mathrm{R}}}}{n\mathrm{R}}=(\mathrm{T}_2)^{1+\frac{\mathrm{\bar{c}}_v}{\mathrm{R}}}
\end{displaymath}
\newpage
sustituyendo y despejando $\mathrm{\bar{c}}_v$ obtenemos

\begin{displaymath}
	\mathrm{T}_2=\left[ \frac{\mathrm{P}_2\mathrm{V}_1\mathrm{T}_1^{\frac{3}{2}}}{n\mathrm{R}} \right]^{\frac{2}{5}}
\end{displaymath}

Sustituyendo los valores

\begin{displaymath}
	\mathrm{T}_2=\left[ \frac{(1\mathrm{atm})(2.463\mathrm{L})(300.15\mathrm{K})^{\frac{3}{2}}}{(1\mathrm{mol})(0.08206\; \frac{\mathrm{atm}\cdot\mathrm{L}}{\mathrm{mol}\cdot\mathrm{K}})} \right]
\end{displaymath}

\begin{displaymath}
	\therefore\mathrm{T}_2= 119.49\;\mathrm{K}
\end{displaymath}

Calculando $\Delta\mathrm{E}$

\begin{displaymath}
	\Delta\mathrm{E}=\mathrm{\bar{c}}_v(\mathrm{T}_2-\mathrm{T}_1)
\end{displaymath}

\begin{displaymath}
	\Delta\mathrm{E}=\frac{3}{2}\left( 0.08206\; \frac{\mathrm{atm}\cdot\mathrm{L}}{\mathrm{mol}\cdot\mathrm{K}} \right) \left(119.49-300.15\right)\mathrm{K}
\end{displaymath}

\begin{displaymath}
	\therefore\Delta\mathrm{E}=-2.26\frac{\mathrm{kJ}}{\mathrm{mol}}
\end{displaymath}

Calculando $\mathrm{W}$

\begin{displaymath}
	\mathrm{W}=-\Delta\mathrm{E}
\end{displaymath}

\begin{displaymath}
	\therefore\mathrm{W}=2.26\frac{\mathrm{kJ}}{\mathrm{mol}}
\end{displaymath}

calculando $\Delta\mathrm{H}$

\begin{displaymath}
	\Delta\mathrm{H}=(\mathrm{\bar{c}}_v+\mathrm{R})(\mathrm{T}_2-\mathrm{T}_1)
\end{displaymath}

\begin{displaymath}
	\Rightarrow\Delta\mathrm{H}=\frac{5\mathrm{R}}{2}(\mathrm{T}_2-\mathrm{T}_1)
\end{displaymath}

\begin{displaymath}
	\Rightarrow\Delta\mathrm{H}=\frac{5}{2}\left( 0.08206\; \frac{\mathrm{atm}\cdot\mathrm{L}}{\mathrm{mol}\cdot\mathrm{K}} \right)\left( 119.49-300.15 \right)\mathrm{K}
\end{displaymath}

\begin{displaymath}
	\therefore\Delta\mathrm{H}=-3.76\frac{\mathrm{kJ}}{\mathrm{mol}}
\end{displaymath}
\newpage
Caso 2 con $\mathrm{\bar{c}}_v=\frac{5\mathrm{R}}{2}$\\
\\
Tomando como base las ecuaciones utilizadas anteriormente, el procedimiento es el mismo pero con el valor de $\mathrm{\bar{c}}_v$ distinto, por lo cual solo se mostraran las ecuaciones finales y el resultado, empezando por la temperatura

\begin{displaymath}
	\mathrm{T}_2=\left[ \frac{\mathrm{P}_2\mathrm{V}_1\mathrm{T}_1^{\frac{5}{2}}}{n\mathrm{R}} \right]^{\frac{2}{7}}
\end{displaymath}

\begin{displaymath}
	\therefore\mathrm{T}_2=155.46\;\mathrm{K}
\end{displaymath}

Para $\Delta\mathrm{E}$

\begin{displaymath}
	\Delta\mathrm{E}=\frac{5R}{2}(\mathrm{T}_2-\mathrm{T}_1)
\end{displaymath}

\begin{displaymath}
	\therefore\Delta\mathrm{E}=-3.01\;\frac{\mathrm{kJ}}{\mathrm{mol}}
\end{displaymath}

Para $\mathrm{W}$

\begin{displaymath}
	\mathrm{W}=-\Delta\mathrm{E}
\end{displaymath}

\begin{displaymath}
	\therefore\mathrm{W}=3.01\;\frac{\mathrm{kJ}}{\mathrm{mol}}
\end{displaymath}

Para $\Delta\mathrm{H}$

\begin{displaymath}
	\Delta\mathrm{H}=\frac{7\mathrm{R}}{2}(\mathrm{T}_2-\mathrm{T}_1)
\end{displaymath}

\begin{displaymath}
	\therefore\Delta\mathrm{H}=-4.22\;\frac{\mathrm{kJ}}{\mathrm{mol}}
\end{displaymath}
\newpage

\section*{Problema}

un mol de gas ideal se comprime adiabaticamente en una sola etapa con una presión constante de oposición a $10\;\mathrm{atm}$. Inicialmente el gas está a $27^{\circ}\mathrm{C}$ y $1\mathrm{atm}$ de presión; la presión final es de $10\mathrm{atm}$. Calcular la temperatura final del gas, $\mathrm{Q}$, $\mathrm{W}$, $\Delta\mathrm{E}$ y $\Delta\mathrm{H}$. Resolver para dos casos:
\begin{itemize}
	\item Caso 1: Gas monoatómico con $\mathrm{\bar{c}}_v=\frac{3\mathrm{R}}{2}$
	
	\item Caso 2: Gas diatómico con $\mathrm{\bar{c}}_v=\frac{5\mathrm{R}}{2}$
\end{itemize}

¿Cómo se afectarían los resultados si se utilizan $n$ moles en vez de un mol?\\
\\

Caso 1 con $\mathrm{\bar{c}}_v=\frac{3\mathrm{R}}{2}$ y un mol

\begin{center}
	\begin{tabular}{| c | c |}
		\hline
		Estado 1 &    Estado 2 \\ \hline
		$\mathrm{T}_1=27^{\circ}\mathrm{C}$\; (300.15$\mathrm{K}$) &    $\mathrm{T}_2=?$ \\
		$\mathrm{P}_1=1\mathrm{atm}$ &    $\mathrm{P}_2=10\mathrm{atm}$ \\
		$\mathrm{V}_1=?$ &    $\mathrm{V}_2=?$\\
		$n=1\mathrm{mol}$&    $n=1\mathrm{mol}$\\ \hline
	\end{tabular}

\begin{displaymath}
	\mathrm{R}=0.08206\; \frac{\mathrm{atm}\cdot\mathrm{L}}{\mathrm{mol}\cdot\mathrm{K}}
\end{displaymath}
\end{center}

Calculando $\mathrm{V}_1$

\begin{displaymath}
	\mathrm{P}_1\mathrm{V}_1=n\mathrm{R}\mathrm{T}_1\Rightarrow \mathrm{V}_1=\frac{n\mathrm{R}\mathrm{T}_1}{\mathrm{P}_1}
\end{displaymath}

\begin{displaymath}
	\mathrm{V}_1=\frac{(1\mathrm{mol})\left(0.08206\; \frac{\mathrm{atm}\cdot\mathrm{L}}{\mathrm{mol}\cdot\mathrm{K}}\right)(300.15\mathrm{K})}{1\mathrm{atm}}
\end{displaymath}

\begin{displaymath}
	\therefore\mathrm{V_1}=24.63\;\mathrm{L}
\end{displaymath}

Ahora para calcular $\mathrm{T}_2$ usamos la siguiente formula

\begin{displaymath}
	\mathrm{P}\Delta\mathrm{V}= n\mathrm{\bar{c}}_v\Delta\mathrm{T}
\end{displaymath}

Desarrollando y sustituyendo $\mathrm{V}_2$ con base en la ecuación encontrada anteriormente

\begin{displaymath}
	n\mathrm{\bar{c}}_v\mathrm{T}_1+\mathrm{P}_2\mathrm{V}_1=\mathrm{T}_2\left[ \mathrm{\bar{c}}_vn+n\mathrm{R} \right]
\end{displaymath}

Sustituyendo el valor de $\mathrm{\bar{c}}_v$ y desarrollando

\begin{displaymath}
	\mathrm{T}_2=\frac{2}{5}\left[ \frac{3\mathrm{T}_1}{2}+\frac{\mathrm{P}_2\mathrm{V}_1}{n\mathrm{R}} \right]
\end{displaymath}

Sustituyendo valores

\begin{displaymath}
	\mathrm{T}_2=\frac{2}{5}\left[ \frac{3(300.15\mathrm{K})}{2}+\frac{(10\mathrm{atm})(24.63\mathrm{L})}{(0.08206\; \frac{\mathrm{atm}\cdot\mathrm{L}}{\mathrm{mol}\cdot\mathrm{K}})(1\mathrm{mol})} \right]
\end{displaymath}

\begin{displaymath}
	\therefore\mathrm{T}_2=1380.67\;\mathrm{K}
\end{displaymath}

Al ser un proceso adiabático se tiene

\begin{displaymath}
	\mathrm{Q}=0
\end{displaymath}

Calculando $\Delta\mathrm{E}$

\begin{displaymath}
	\Delta\mathrm{E}= \mathrm{\bar{c}}_v(\mathrm{T}_2-\mathrm{T}_1) 
\end{displaymath}

\begin{displaymath}
	\Rightarrow \Delta\mathrm{E}=\frac{3}{2}\left( 0.08206\; \frac{\mathrm{atm}\cdot\mathrm{L}}{\mathrm{mol}\cdot\mathrm{K}} \right)\left( 1380.67-300.15 \right)\mathrm{K}
\end{displaymath}

\begin{displaymath}
	\therefore\Delta\mathrm{E}=13.5\;\frac{\mathrm{kJ}}{\mathrm{mol}}
\end{displaymath}

Calculando $\mathrm{W}$

\begin{displaymath}
	\Delta\mathrm{E}=-\mathrm{W}\Rightarrow\mathrm{W}=-13.5\frac{kJ}{\mathrm{mol}}
\end{displaymath}

Calculando $\Delta\mathrm{H}$

\begin{displaymath}
	\Delta\mathrm{H}=(\mathrm{\bar{c}}_v+\mathrm{R})(\mathrm{T}_2-\mathrm{T}_1)
\end{displaymath}

\begin{displaymath}
	\Rightarrow\Delta\mathrm{H}=\frac{5\mathrm{R}}{2}(\mathrm{T}_2-\mathrm{T}_1)
\end{displaymath}

\begin{displaymath}
	\Rightarrow\Delta\mathrm{H}=\frac{5}{2}\left( 0.08206\; \frac{\mathrm{atm}\cdot\mathrm{L}}{\mathrm{mol}\cdot\mathrm{K}} \right)(1380.67-300.15)\mathrm{K}
\end{displaymath}

\begin{displaymath}
	\therefore\Delta\mathrm{H}=22.4\;\frac{\mathrm{kJ}}{\mathrm{mol}}
\end{displaymath}

Caso 2 $\mathrm{\bar{c}}_v=\frac{5\mathrm{R}}{2}$ y un mol:\\
\\
Para este caso, las ecuaciones anteriores son iguales, solo se sustituirá el valor de $\mathrm{\bar{c}}_v$, por lo que solo se presentarán las ecuaciones con la sustitución y desarrollo final, asi como el resultado obtenido de sustituir los valores del estado.\\
\\
El volumen es el mismo, por lo que es:

\begin{displaymath}
	\mathrm{V}_1=24.63\mathrm{L}
\end{displaymath}

Calculando $\mathrm{T}_2$

\begin{displaymath}
	\mathrm{T}_2=\frac{2}{7}\left[ \frac{5\mathrm{T}_1}{2}+\frac{\mathrm{P}_2\mathrm{V}_1}{n\mathrm{R}} \right]
\end{displaymath}

\begin{displaymath}
	\therefore\mathrm{T}_2=1071.95\;\mathrm{K}
\end{displaymath}

Al ser proceso adiabático entonces

\begin{displaymath}
	\mathrm{Q}=0
\end{displaymath}

Calculando $\Delta\mathrm{E}$

\begin{displaymath}
	\Delta\mathrm{E}=\frac{5\mathrm{R}}{2}(\mathrm{T}_2-\mathrm{T}_1)
\end{displaymath}

\begin{displaymath}
	\therefore\Delta\mathrm{E}=16\;\frac{\mathrm{kJ}}{\mathrm{mol}}
\end{displaymath}

Calculando $\mathrm{W}$

\begin{displaymath}
	\Delta\mathrm{E}=-\mathrm{W}\Rightarrow \mathrm{W}=-16\;\frac{\mathrm{kJ}}{\mathrm{mol}}
\end{displaymath}

Calculando $\Delta\mathrm{H}$

\begin{displaymath}
	\Delta\mathrm{H}=\frac{7\mathrm{R}}{2}(\mathrm{T}_2-\mathrm{T}_1)
\end{displaymath}

\begin{displaymath}
	\therefore\Delta\mathrm{H}=22.4\;\frac{\mathrm{kJ}}{\mathrm{mol}}
\end{displaymath}

\newpage

Contestando para $n$ moles\\
\\
Debido a que el resto de datos son exactamente iguales podemos determinar que el valor que se obtuvo de volumen será proporcional al número de moles quedando de la siguiente forma para el primer caso:

\begin{displaymath}
	\mathrm{V}_1=24.63\;n
\end{displaymath}

Ahora para la temperatura, haciendo un análisis de la ecuación, podemos observar que los valores de moles se cancelan, por lo que podemos decir que la temperatura no depende de la cantidad de moles que se tengan, lo cual aplica para ambos casos.\\
\\
Para $\mathrm{Q}$, $\mathrm{W}$, $\Delta\mathrm{E}$ y $\Delta\mathrm{H}$ pasa exactamente lo mismo que con el volumen, son proporcionales a la cantidad de moles que se tenga, por lo que en los casos que ya se presentaron anteriormente bastará con multiplicar la cantidad deseada por el número de moles que se quiera usar.













\end{document}